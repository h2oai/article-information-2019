%  LaTeX support: latex@mdpi.com 
%  In case you need support, please attach all files that are necessary for compiling as well as the log file, and specify the details of your LaTeX setup (which operating system and LaTeX version / tools you are using).

%=================================================================
\documentclass[information,article,submit,moreauthors,pdftex]{Definitions/mdpi} 

% If you would like to post an early version of this manuscript as a preprint, you may use preprint as the journal and change 'submit' to 'accept'. The document class line would be, e.g., \documentclass[preprints,article,accept,moreauthors,pdftex]{mdpi}. This is especially recommended for submission to arXiv, where line numbers should be removed before posting. For preprints.org, the editorial staff will make this change immediately prior to posting.

%--------------------
% Class Options:
%--------------------
%----------
% journal
%----------
% Choose between the following MDPI journals:
% acoustics, actuators, addictions, admsci, aerospace, agriculture, agriengineering, agronomy, algorithms, animals, antibiotics, antibodies, antioxidants, applsci, arts, asc, asi, atmosphere, atoms, axioms, batteries, bdcc, behavsci , beverages, bioengineering, biology, biomedicines, biomimetics, biomolecules, biosensors, brainsci , buildings, cancers, carbon , catalysts, cells, ceramics, challenges, chemengineering, chemistry, chemosensors, children, cleantechnol, climate, clockssleep, cmd, coatings, colloids, computation, computers, condensedmatter, cosmetics, cryptography, crystals, dairy, data, dentistry, designs , diagnostics, diseases, diversity, drones, econometrics, economies, education, ejihpe, electrochem, electronics, energies, entropy, environments, epigenomes, est, fermentation, fibers, fire, fishes, fluids, foods, forecasting, forests, fractalfract, futureinternet, futurephys, galaxies, games, gastrointestdisord, gels, genealogy, genes, geohazards, geosciences, geriatrics, hazardousmatters, healthcare, heritage, highthroughput, horticulturae, humanities, hydrology, ijerph, ijfs, ijgi, ijms, ijns, ijtpp, informatics, information, infrastructures, inorganics, insects, instruments, inventions, iot, j, jcdd, jcm, jcp, jcs, jdb, jfb, jfmk, jimaging, jintelligence, jlpea, jmmp, jmse, jnt, jof, joitmc, jpm, jrfm, jsan, land, languages, laws, life, literature, logistics, lubricants, machines, magnetochemistry, make, marinedrugs, materials, mathematics, mca, medicina, medicines, medsci, membranes, metabolites, metals, microarrays, micromachines, microorganisms, minerals, modelling, molbank, molecules, mps, mti, nanomaterials, ncrna, neuroglia, nitrogen, notspecified, nutrients, ohbm, optics, particles, pathogens, pharmaceuticals, pharmaceutics, pharmacy, philosophies, photonics, physics, plants, plasma, polymers, polysaccharides, preprints , proceedings, processes, proteomes, psych, publications, quantumrep, quaternary, qubs, reactions, recycling, religions, remotesensing, reports, resources, risks, robotics, safety, sci, scipharm, sensors, separations, sexes, signals, sinusitis, smartcities, sna, societies, socsci, soilsystems, sports, standards, stats, surfaces, surgeries, sustainability, symmetry, systems, technologies, test, toxics, toxins, tropicalmed, universe, urbansci, vaccines, vehicles, vetsci, vibration, viruses, vision, water, wem, wevj

%---------
% article
%---------
% The default type of manuscript is "article", but can be replaced by: 
% abstract, addendum, article, benchmark, book, bookreview, briefreport, casereport, changes, comment, commentary, communication, conceptpaper, conferenceproceedings, correction, conferencereport, expressionofconcern, extendedabstract, meetingreport, creative, datadescriptor, discussion, editorial, essay, erratum, hypothesis, interestingimages, letter, meetingreport, newbookreceived, obituary, opinion, projectreport, reply, retraction, review, perspective, protocol, shortnote, supfile, technicalnote, viewpoint
% supfile = supplementary materials

%----------
% submit
%----------
% The class option "submit" will be changed to "accept" by the Editorial Office when the paper is accepted. This will only make changes to the frontpage (e.g., the logo of the journal will get visible), the headings, and the copyright information. Also, line numbering will be removed. Journal info and pagination for accepted papers will also be assigned by the Editorial Office.

%------------------
% moreauthors
%------------------
% If there is only one author the class option oneauthor should be used. Otherwise use the class option moreauthors.

%---------
% pdftex
%---------
% The option pdftex is for use with pdfLaTeX. If eps figures are used, remove the option pdftex and use LaTeX and dvi2pdf.

%=================================================================
\firstpage{1} 
\makeatletter 
\setcounter{page}{\@firstpage} 
\makeatother
\pubvolume{xx}
\issuenum{1}
\articlenumber{5}
\pubyear{2019}
\copyrightyear{2019}
%\externaleditor{Academic Editor: name}
\history{Received: date; Accepted: date; Published: date}
%\updates{yes} % If there is an update available, un-comment this line

%% MDPI internal command: uncomment if new journal that already uses continuous page numbers 
%\continuouspages{yes}

%------------------------------------------------------------------
% The following line should be uncommented if the LaTeX file is uploaded to arXiv.org
%\pdfoutput=1

%=================================================================
% Add packages and commands here. The following packages are loaded in our class file: fontenc, calc, indentfirst, fancyhdr, graphicx, lastpage, ifthen, lineno, float, amsmath, setspace, enumitem, mathpazo, booktabs, titlesec, etoolbox, amsthm, hyphenat, natbib, hyperref, footmisc, geometry, caption, url, mdframed, tabto, soul, multirow, microtype, tikz

%=================================================================
%% Please use the following mathematics environments: Theorem, Lemma, Corollary, Proposition, Characterization, Property, Problem, Example, ExamplesandDefinitions, Hypothesis, Remark, Definition, Notation, Assumption
%% For proofs, please use the proof environment (the amsthm package is loaded by the MDPI class).

%=================================================================
% Full title of the paper (Capitalized)
\Title{Responsible Machine Learning\\\large{with Interpretable Models, Post-hoc Explanation, and Disparate Impact Testing}}

% Author Orchid ID: enter ID or remove command
\newcommand{\orcidauthorA}{0000-0000-000-000X} % Add \orcidA{} behind the author's name
%\newcommand{\orcidauthorB}{0000-0000-000-000X} % Add \orcidB{} behind the author's name

% Authors, for the paper (add full first names)
\Author{Navdeep Gill $^{1, \ddagger}$, Patrick Hall $^{1,\ddagger,*}$, Kim Montgomery $^{1,\ddagger}$, and Nicholas Schmidt $^{2, \ddagger}$}

% Authors, for metadata in PDF
\AuthorNames{Navdeep Gill, Patrick Hall, Kim Montgomery, and Nicholas Schmidt}

% Affiliations / Addresses (Add [1] after \address if there is only one affiliation.)
\address{%
$^{1}$ \quad H2O.ai\\
$^{2}$ \quad BLDS, LLC}

% Contact information of the corresponding author
\corres{Correspondence: phall@h2o.ai; nschmidt@bldsllc.com}

% Current address and/or shared authorship
%\firstnote{Current address: Affiliation 3} 
\secondnote{These authors contributed equally to this work.}
% The commands \thirdnote{} till \eighthnote{} are available for further notes

%\simplesumm{} % Simple summary

%\conference{} % An extended version of a conference paper

% Abstract (Do not insert blank lines, i.e. \\) 
\abstract{This text outlines a viable approach for training and evaluating complex machine learning systems for high-stakes, human-centered, or regulated applications using common Python programming tools. The accuracy and intrinsic interpretability of two types of constrained models, monotonic gradient boosting machines (M-GBM) and explainable neural networks (XNN), a deep learning architecture well-suited for structured data, are assessed on simulated datasets with known feature importance and sociological bias characteristics and on realistic, publicly available example datasets. For maximum transparency and the potential generation of personalized adverse action notices, the constrained models are analyzed using post-hoc explanation techniques including plots of individual conditional expectation (ICE) and global and local gradient-based or Shapley feature importance. The constrained model predictions are also tested for disparate impact and other types of sociological bias using straightforward group fairness measures. By combining innovations in interpretable models, post-hoc explanation, and bias testing with accessible software tools, this text aims to provide a template workflow for important machine learning applications that require high accuracy and interpretability and low disparate impact.}

% Keywords
\keyword{Machine Learning; Neural Network; Gradient Boosting Machine; Interpretable; Explanation; Fairness; Disparate Impact; Python}

% The fields PACS, MSC, and JEL may be left empty or commented out if not applicable
%\PACS{J0101}
%\MSC{}
%\JEL{}

%%%%%%%%%%%%%%%%%%%%%%%%%%%%%%%%%%%%%%%%%%
% Only for the journal Diversity
%\LSID{\url{http://}}

%%%%%%%%%%%%%%%%%%%%%%%%%%%%%%%%%%%%%%%%%%
% Only for the journal Applied Sciences:
%\featuredapplication{Authors are encouraged to provide a concise description of the specific application or a potential application of the work. This section is not mandatory.}
%%%%%%%%%%%%%%%%%%%%%%%%%%%%%%%%%%%%%%%%%%

%%%%%%%%%%%%%%%%%%%%%%%%%%%%%%%%%%%%%%%%%%
% Only for the journal Data:
%\dataset{DOI number or link to the deposited data set in cases where the data set is published or set to be published separately. If the data set is submitted and will be published as a supplement to this paper in the journal Data, this field will be filled by the editors of the journal. In this case, please make sure to submit the data set as a supplement when entering your manuscript into our manuscript editorial system.}

%\datasetlicense{license under which the data set is made available (CC0, CC-BY, CC-BY-SA, CC-BY-NC, etc.)}

%%%%%%%%%%%%%%%%%%%%%%%%%%%%%%%%%%%%%%%%%%
% Only for the journal Toxins
%\keycontribution{The breakthroughs or highlights of the manuscript. Authors can write one or two sentences to describe the most important part of the paper.}

%\setcounter{secnumdepth}{4}
%%%%%%%%%%%%%%%%%%%%%%%%%%%%%%%%%%%%%%%%%%
\begin{document}
%%%%%%%%%%%%%%%%%%%%%%%%%%%%%%%%%%%%%%%%%%

%%%%%%%%%%%%%%%%%%%%%%%%%%%%%%%%%%%%%%%%%%
\setcounter{section}{-1} 
%% Remove this when starting to work on the template.
%\section{How to Use this Template}
%The template details the sections that can be used in a manuscript. Note that the order and names of article sections may differ from the requirements of the journal (e.g., the positioning of the Materials and Methods section). Please check the instructions for authors page of the journal to verify the correct order and names. For any questions, please contact the editorial office of the journal or support@mdpi.com. For LaTeX related questions please contact latex@mdpi.com.
%The order of the section titles is: Introduction, Materials and Methods, Results, Discussion, Conclusions for these journals: aerospace,algorithms,antibodies,antioxidants,atmosphere,axioms,biomedicines,carbon,crystals,designs,diagnostics,environments,fermentation,fluids,forests,fractalfract,informatics,information,inventions,jfmk,jrfm,lubricants,neonatalscreening,neuroglia,particles,pharmaceutics,polymers,processes,technologies,viruses,vision

%%%%%%%%%%%%%%%%%%%%%%%%%%%%%%%%%%%%%%%%%%
\section{Introduction}

%%%%%%%%%%%%%%%%%%%%%%%%%%%%%%%%%%%%%%%%%%
\section{Materials and Methods}

%%%%%%%%%%%%%%%%%%%%%%%%%%%%%%%%%%%%%%%%%%
\subsection{Notation}

To facilitate descriptions of data, modeling, explanatory, and social bias techniques, notation for input and output spaces, datasets, and models is defined.

\subsubsection{Spaces} 
 
	\begin{itemize}
		\item Input features come from the set $\mathcal{X}$ contained in a \textit{P}-dimensional input space, $\mathcal{X} \subset \mathbb{R}^P$.  An arbitrary, potentially unobserved, or future instance of $\mathcal{X}$ is denoted $\mathbf{x}$, $\mathbf{x} \in \mathcal{X}$.
		\item Labels corresponding to instances of $\mathcal{X}$ come from the set $\mathcal{Y}$.
		\item Learned output responses come from the set $\mathcal{\hat{Y}}$. % For regression models, the set $\mathcal{\hat{Y}}_r$ is also contained in a $C$-dimensional output space, $\mathcal{\hat{Y}}_r \subset \mathbb{R}^{C_r}$. For classification models, the set $\mathcal{\hat{Y}}_c$ typically contains a column vector for each unique class in $\mathcal{Y}$. Hence, $\mathcal{\hat{Y}}_c$ is contained in a $C'$-dimensional output space,  $\mathcal{\hat{Y}}_c \subset \mathbb{R}^{C'_c}$.
	\end{itemize}	
	
\subsubsection{Datasets} 

	\begin{itemize}
		\item The input dataset $\mathbf{X}$ is composed of observed instances of the set $\mathcal{X}$ with a corresponding dataset of labels $\mathbf{Y}$, observed instances of the set $\mathcal{Y}$. 
		\item Each $i$-th observation of $\mathbf{X}$ is denoted as $\mathbf{x}^{(i)} = $  
		$[x_0^{(i)}, x_1^{(i)}, \dots, x_{\textit{P}-1}^{(i)}]$, with corresponding $i$-th labels in $\mathbf{Y}, \mathbf{y}^{(i)}$, and corresponding predictions in $\mathbf{\hat{Y}}, \mathbf{\hat{y}}^{(i)}$. % = [y_0^{(i)}, y_1^{(i)}, \dots, y_{\textit{C}-1}^{(i)}]$.
		\item $\mathbf{X}$ and $\mathbf{Y}$ consist of $N$ tuples of observations: $[(\mathbf{x}^{(0)},\mathbf{y}^{(0)}), (\mathbf{x}^{(1)},\mathbf{y}^{(1)}), \dots,(\mathbf{x}^{(N-1)},\mathbf{y}^{(N-1)})]$. %\\ $\mathbf{x}^{(i)} \in \mathcal{X}$, $\mathbf{y}^{(i)} \in \mathcal{Y}$.
		\item Each $j$-th input column vector of $\mathbf{X}$ is denoted as $X_j = [x_{j}^{(0)}, x_{j}^{(1)}, \dots, x_{j}^{(N-1)}]^T$.
	\end{itemize}	 

\subsubsection{Models}

	\begin{itemize}
		\item A type of machine learning model $g$, selected from a hypothesis set $\mathcal{H}$, is trained to represent an unknown signal-generating function $f$ observed as  $\mathbf{X}$ with labels $\mathbf{Y}$ using a training algorithm $\mathcal{A}$: 
		$ \mathbf{X}, \mathbf{Y} \xrightarrow{\mathcal{A}} g$, such that $g \approx f$.
		\item $g$ generates learned output responses on the input dataset $g(\mathbf{X}) = \mathbf{\hat{Y}}$, and on the general input space $g(\mathcal{X}) = \mathcal{\hat{Y}}$.
		\item The model to be explained is denoted as $g$.
	\end{itemize}

%%%%%%%%%%%%%%%%%%%%%%%%%%%%%%%%%%%%%%%%%%
\subsection{Data Description}

%%%%%%%%%%%%%%%%%%%%%%%%%%%%%%%%%%%%%%%%%%
\subsection{Model Description}

\subsubsection{Explainable Neural Network}

\subsubsection{Monotonically Constrained Gradient Boosting Machine}

Monotonic gradient boosting machines (M-GBMs) constrain typical GBM training to consider only tree splits that obey user-defined positive and negative monotonicity constraints. The M-GBM remains an additive combination of many trees, $T_b$, but with a set of splitting rules that respect the monotonicity constraints,  $\Theta^\text{mono}_b$.

\begin{equation}
\begin{aligned}\label{eq:rf}
g^{\text{mono}}(\mathbf{x}) &= \sum_{b=1}^B T_b\left(\mathbf{x}; \Theta^\text{mono}_b\right)
\end{aligned}
\end{equation}

\noindent As in unconstrained GBM, $\Theta_b$ is selected in a greedy, additive fashion by minimizing a regularized loss function that considers known target labels, $\mathbf{y}$, the predictions of all subsequently trained trees in the M-GBM, $g^{\text{mono}}_{b-1}(\mathbf{X})$, and a regularization term that penalizes complexity in the current tree, $\Omega(T_b)$. For each $i$-th observation  and the $b$-th iteration the loss function, $\mathcal{L}_{i,b}$, can be defined as:

\begin{equation}
\begin{aligned}
\mathcal{L}_{i,b} =\sum_{i=0}^{N-1}l(y^{(i)}, g^{\text{mono}}_{b-1}(\mathbf{x}^{(i)}), T_b(\mathbf{x}^{(i)};\Theta^\text{mono}_b)) + \Omega(T_b)\\
\end{aligned}
\end{equation}

\noindent In addition to $\mathcal{L}_{i,b}$, $\Theta^\text{mono}_b$ is constrained by applying additional splitting rules for each binary split rule, $\theta_{b,j,k} \in \Theta_b$. Each $\theta_{b,j,k}$ is associated with a feature, $X_j$, and can be the $k$-th such split associated with $X_j$ in $T_b$. Each $\theta_{b,j,k}$ also results in left and right child nodes with a numeric weight, $\{w_{b,j,k,L}, w_{b,j,kR}\}$. For terminal nodes, each $w_{b,j,k}$ is essentially the model prediction. For two values of some feature $X_j$, $x^{\alpha}_j \le x^{\beta}_j$, where the prediction for each value results in $T_b(x^{\alpha}_j; \Theta_b) = w_\alpha$ and $T_b(x^{\beta}_j; \Theta_b) = w_\beta$, $\Theta_b$ is said to be positive monotonic if:

\begin{enumerate}
\item For the first and highest split in $T_b$ involving $X_j$, any $\theta_{b,j,0}$ causing the left child weight to be greater than the right child weight, $T(x_j; \theta_{b,j,0}) = \{w_{b,j,0,L}, w_{j,0,R}\}$ where $w_{b,j,0,L} > w_{b,j,0,R}$ is not considered. 
\item For any subsequent left child node involving $X_j$, any $\theta_{b,j, 1+}$ causing $T(x_j; \theta_{b,j,1+}) = \{w_{b,j,1+,L}, w_{b,j,1+R}\}$ where $w_{b,j,1+,L} > w_{b,j,1+,R}$ is not considered.
\item Moreover, for any subsequent left child node involving $X_j$, $T(x_j; \theta_{b,j,k+}) = \{w_{b,j,k,L}, w_{b,j,k,R}\}$ for $k > 0$, $\{w_{b,j,k,L}, w_{b,j,k,R}\}$ are bound by the preceding set of $\{w_{b,j,k-1,L}, w_{b,j,k-1, R}\}$ such that $ \{w_{b,j,k,L}, w_{b,j,k,R}\} \le \frac{w_{b,j,k-1,L} + w_{b,j,k-1,R}}{2}$.
\item (1) and (2) are also applied to all right child nodes, except that for right child nodes $\{w_{b,j,k,L}, w_{b,j,k,R}\} \ge \frac{w_{b,j,k-1,L} + w_{b,j,k-1,R}}{2}$.

\end{enumerate}

\noindent Note that for any $X_j$ in any $g_{\text{mono}}~T_b$ left subtrees will alway produce lower predictions than right subtrees, and that any $g_{\text{mono}}(\mathbf{x})$ is a sequential addition of each $T_b$ output, with the application of a monotonic logit or softmax link function for classifications. Also note that each tree's root node will always obey monotonicity constraints, as $T(x^{\alpha}_j; \theta_{b,0}) = T(x^{\beta}_j; \theta_{b,0})$, ensuring $T(x^{\alpha}_j; \theta_{b,j,0}) = w_{b,j,0,L} \le T(x^{\beta}_j; \theta_{b,j,0}) = w_{b,j,0,R}$. For negative monotonic constraints left and right splitting rules are switched, and tree pruning strategies can be applied.

% Tree pruning?

%%%%%%%%%%%%%%%%%%%%%%%%%%%%%%%%%%%%%%%%%%
\subsection{Explanatory Method Description}

\subsubsection{Partial Dependence and Individual Conditional Expectation}

Partial dependence (PD) plots are a widely-used method for describing the average predictions of a complex model $g$ across some partition of data $\mathbf{X}$ for some interesting input feature $X_j$ \cite{esl}. Individual conditional expectation (ICE) plots are a newer method that describes the local behavior of $g$ for a single instance $\mathbf{x} \in \mathcal{X}$. Partial dependence and ICE can be combined in the same plot to compensate for known weaknesses of partial dependence, to identify interactions modeled by $g$, and to create a holistic portrait of the predictions of a complex model for some $X_j$  \cite{ice_plots}.

Following \citet{esl} a single feature $X_j \in \mathbf{X}$ and its complement set $\mathbf{X}_{(-j)} \in \mathbf{X}$ (where $X_j \cup \mathbf{X}_{(-j)} = \mathbf{X}$) is considered. $\text{PD}(X_j, g)$ for a given feature $X_j$ is estimated as the average output of the learned function $g(\mathbf{X})$ when all the observations of $X_j$ are set to a constant $x \in \mathcal{X}$ and $\mathbf{X}_{(-j)}$ is left unchanged. $\text{ICE}(x_j, \mathbf{x}, g)$ for a given instance $\mathbf{x}$ and feature $x_j$ is estimated as the output of $g(\mathbf{x})$ when $x_j$ is set to a constant $x \in \mathcal{X}$ and all other features $\mathbf{x} \in \mathbf{X}_{(-j)}$ are left untouched. Partial dependence and ICE curves are usually plotted over some set of constants $x \in \mathcal{X}$. 

\subsubsection{Shapley Values}

Shapley explanations, including Tree SHAP (SHapley Additive exPlanations) , are a class of additive, locally accurate feature contribution measures with long-standing theoretical support \cite{shapley}. Shapley explanations are the only possible locally accurate and globally consistent feature contribution values, meaning that Shapley explanation values for input features always sum to $g(\mathbf{x})$ and that Shapley explanation values can never decrease for some $x_j$ when $g$ is changed such that $x_j$ truly makes a stronger contribution to $g(\mathbf{x})$ \cite{shapley}. 

For some observation $\mathbf{x} \in \mathcal{X}$, Shapley explanations take the form:

\begin{equation}
\label{eq:shap_additive}
\begin{aligned}
g(\mathbf{x}) = \phi_0 + \sum_{j=0}^{j=\mathcal{P} - 1} \phi_j \mathbf{z}_j
\end{aligned}
\end{equation}

\noindent In Equation \ref{eq:shap_additive}, $\mathbf{z} \in \{0,1\}^\mathcal{P}$ is a binary representation of $\mathbf{x}$ where 0 indicates missingness. Each $\phi_j$ is the local feature contribution value associated with $x_j$ and $\phi_0$ is the average of $g(\mathbf{X})$. 

Shapley values can be estimated in different ways. Tree SHAP is a specific implementation of Shapley explanations that relies on traversing internal tree structures to estimate the impact of each $x_j$ for some $g(\mathbf{x})$ of interest \cite{tree_shap}.

\begin{equation}
\label{eq:shap_contrib}
\begin{aligned}
\phi_{j} = \sum_{S \subseteq \mathcal{P} \setminus \{j\}}\frac{|S|!(\mathcal{P} -|S| -1)!}{\mathcal{P}!}[g_x(S \cup \{j\}) - g_x(S)]
\end{aligned}
\end{equation}

%%%%%%%%%%%%%%%%%%%%%%%%%%%%%%%%%%%%%%%%%%
\subsection{Social Bias Test Description}

%%%%%%%%%%%%%%%%%%%%%%%%%%%%%%%%%%%%%%%%%%
\subsection{Software Resources}
% Description of code

%%%%%%%%%%%%%%%%%%%%%%%%%%%%%%%%%%%%%%%%%%
\section{Results}

%%%%%%%%%%%%%%%%%%%%%%%%%%%%%%%%%%%%%%%%%%
\subsection{Simulated Data Results}

%%%%%%%%%%%%%%%%%%%%%%%%%%%%%%%%%%%%%%%%%%
\subsection{Loan Data Results}

%%%%%%%%%%%%%%%%%%%%%%%%%%%%%%%%%%%%%%%%%%
\section{Discussion}
% Bivariate monotonicity constraints likely dampen impact of interactions

%%%%%%%%%%%%%%%%%%%%%%%%%%%%%%%%%%%%%%%%%%
\section{Conclusions}
% Model debugging
% Model management and monitoring
% Appeal

%%%%%%%%%%%%%%%%%%%%%%%%%%%%%%%%%%%%%%%%%%
\vspace{6pt} 


%%%%%%%%%%%%%%%%%%%%%%%%%%%%%%%%%%%%%%%%%%
\authorcontributions{, N.G.; , P.H.; , K.M.; , N.S.}%For research articles with several authors, a short paragraph specifying their individual contributions must be provided. The following statements should be used ``conceptualization, X.X. and Y.Y.; methodology, X.X.; software, X.X.; validation, X.X., Y.Y. and Z.Z.; formal analysis, X.X.; investigation, X.X.; resources, X.X.; data curation, X.X.; writing--original draft preparation, X.X.; writing--review and editing, X.X.; visualization, X.X.; supervision, X.X.; project administration, X.X.; funding acquisition, Y.Y.'', please turn to the  \href{http://img.mdpi.org/data/contributor-role-instruction.pdf}{CRediT taxonomy} for the term explanation. Authorship must be limited to those who have contributed substantially to the work reported.}

%%%%%%%%%%%%%%%%%%%%%%%%%%%%%%%%%%%%%%%%%%
\funding{This research received no external funding.}

%%%%%%%%%%%%%%%%%%%%%%%%%%%%%%%%%%%%%%%%%%
\acknowledgments{Wen Phan for work in formalizing our notation.}

%%%%%%%%%%%%%%%%%%%%%%%%%%%%%%%%%%%%%%%%%%
\conflictsofinterest{} 

%%%%%%%%%%%%%%%%%%%%%%%%%%%%%%%%%%%%%%%%%%
%% optional
\abbreviations{The following abbreviations are used in this manuscript:\\}

%%%%%%%%%%%%%%%%%%%%%%%%%%%%%%%%%%%%%%%%%%
\reftitle{References}

%=====================================
% References, variant A: external bibliography
%=====================================
\externalbibliography{yes}
\bibliography{article-information-2019}

%%%%%%%%%%%%%%%%%%%%%%%%%%%%%%%%%%%%%%%%%%
\end{document}


%%%%%%%%%%%%%%%%%%%%%%%%%%%%%%%%%%%%%%%%%%

%%%%%%%%%%%%%%%%%%%%%%%%%%%%%%%%%%%%%%%%%%

%%%%%%%%%%%%%%%%%%%%%%%%%%%%%%%%%%%%%%%%%%


%This section may be divided by subheadings. It should provide a concise and precise description of the experimental results, their interpretation as well as the experimental conclusions that can be drawn.
%\begin{quote}
%This section may be divided by subheadings. It should provide a concise and precise description of the experimental results, their interpretation as well as the experimental conclusions that can be drawn.
%\end{quote}



%%%%%%%%%%%%%%%%%%%%%%%%%%%%%%%%%%%%%%%%%%
%\subsection{Subsection}
%\unskip
%\subsubsection{Subsubsection}

%Bulleted lists look like this:
%\begin{itemize}[leftmargin=*,labelsep=5.8mm]
%\item	First bullet
%\item	Second bullet
%\item	Third bullet
%\end{itemize}

%Numbered lists can be added as follows:
%\begin{enumerate}[leftmargin=*,labelsep=4.9mm]
%\item	First item 
%\item	Second item
%\item	Third item
%\end{enumerate}

%The text continues here.

%\subsection{Figures, Tables and Schemes}

%All figures and tables should be cited in the main text as Figure 1, Table 1, etc.

%\begin{figure}[H]
%\centering
%\includegraphics[width=2 cm]{Definitions/logo-mdpi}
%\caption{This is a figure, Schemes follow the same formatting. If there are multiple panels, they should be listed as: (\textbf{a}) Description of what is contained in the first panel. (\textbf{b}) Description of what is contained in the second panel. Figures should be placed in the main text near to the first time they are cited. A caption on a single line should be centered.}
%\end{figure}   
 
%Text

%Text

%\begin{table}[H]
%\caption{This is a table caption. Tables should be placed in the main text near to the first time they are cited.}
%\centering
%% \tablesize{} %% You can specify the fontsize here, e.g., \tablesize{\footnotesize}. If commented out \small will be used.
%\begin{tabular}{ccc}
%\toprule
%\textbf{Title 1}	& \textbf{Title 2}	& \textbf{Title 3}\\
%\midrule
%entry 1		& data			& data\\
%entry 2		& data			& data\\
%\bottomrule
%\end{tabular}
%\end{table}

%Text

%Text

%\begin{listing}[H]
%\caption{Title of the listing}
%\rule{\textwidth}{1pt}
%\raggedright Text of the listing. In font size footnotesize, small, or normalsize. Preferred format: left aligned and single spaced. Preferred border format: top border line and bottom border line.
%\rule{\textwidth}{1pt}
%\end{listing}


%\subsection{Formatting of Mathematical Components}

%This is an example of an equation:

%\begin{equation}
%a + b = c
%\end{equation}
%% If the documentclass option "submit" is chosen, please insert a blank line before and after any math environment (equation and eqnarray environments). This ensures correct linenumbering. The blank line should be removed when the documentclass option is changed to "accept" because the text following an equation should not be a new paragraph. 

%Please punctuate equations as regular text. Theorem-type environments (including propositions, lemmas, corollaries etc.) can be formatted as follows:
%% Example of a theorem:
%\begin{Theorem}
%Example text of a theorem.
%\end{Theorem}

%The text continues here. Proofs must be formatted as follows:

%% Example of a proof:
%\begin{proof}[Proof of Theorem 1]
%Text of the proof. Note that the phrase `of Theorem 1' is optional if it is clear which theorem is being referred to.
%\end{proof}
%The text continues here.

%%%%%%%%%%%%%%%%%%%%%%%%%%%%%%%%%%%%%%%%%%
%\section{Discussion}

%Authors should discuss the results and how they can be interpreted in perspective of previous studies and of the working hypotheses. The findings and their implications should be discussed in the broadest context possible. Future research directions may also be highlighted.

%%%%%%%%%%%%%%%%%%%%%%%%%%%%%%%%%%%%%%%%%%
%\section{Materials and Methods}

%Materials and Methods should be described with sufficient details to allow others to replicate and build on published results. Please note that publication of your manuscript implicates that you must make all materials, data, computer code, and protocols associated with the publication available to readers. Please disclose at the submission stage any restrictions on the availability of materials or information. New methods and protocols should be described in detail while well-established methods can be briefly described and appropriately cited.

%Research manuscripts reporting large datasets that are deposited in a publicly available database should specify where the data have been deposited and provide the relevant accession numbers. If the accession numbers have not yet been obtained at the time of submission, please state that they will be provided during review. They must be provided prior to publication.

%Interventionary studies involving animals or humans, and other studies require ethical approval must list the authority that provided approval and the corresponding ethical approval code. 

%%%%%%%%%%%%%%%%%%%%%%%%%%%%%%%%%%%%%%%%%%
%\section{Conclusions}

%This section is not mandatory, but can be added to the manuscript if the discussion is unusually long or complex.

%%%%%%%%%%%%%%%%%%%%%%%%%%%%%%%%%%%%%%%%%%
%\section{Patents}
%This section is not mandatory, but may be added if there are patents resulting from the work reported in this manuscript.

%%%%%%%%%%%%%%%%%%%%%%%%%%%%%%%%%%%%%%%%%%
%% optional
%\supplementary{The following are available online at \linksupplementary{s1}, Figure S1: title, Table S1: title, Video S1: title.}

% Only for the journal Methods and Protocols:
% If you wish to submit a video article, please do so with any other supplementary material.
% \supplementary{The following are available at \linksupplementary{s1}, Figure S1: title, Table S1: title, Video S1: title. A supporting video article is available at doi: link.}


%\noindent 
%\begin{tabular}{@{}ll}
%MDPI & Multidisciplinary Digital Publishing Institute\\
%DOAJ & Directory of open access journals\\
%TLA & Three letter acronym\\
%LD & linear dichroism
%\end{tabular}}

%%%%%%%%%%%%%%%%%%%%%%%%%%%%%%%%%%%%%%%%%%
%% optional
%\appendixtitles{no} %Leave argument "no" if all appendix headings stay EMPTY (then no dot is printed after "Appendix A"). If the appendix sections contain a heading then change the argument to "yes".
%\appendix
%\section{}
%\unskip
%\subsection{}
%The appendix is an optional section that can contain details and data supplemental to the main text. For example, explanations of experimental details that would disrupt the flow of the main text, but nonetheless remain crucial to understanding and reproducing the research shown; figures of replicates for experiments of which representative data is shown in the main text can be added here if brief, or as Supplementary data. Mathematical proofs of results not central to the paper can be added as an appendix.

%\section{}
%All appendix sections must be cited in the main text. In the appendixes, Figures, Tables, etc. should be labeled starting with `A', e.g., Figure A1, Figure A2, etc. 

% Please provide either the correct journal abbreviation (e.g. according to the “List of Title Word Abbreviations” http://www.issn.org/services/online-services/access-to-the-ltwa/) or the full name of the journal.
% Citations an


%=====================================
% References, variant B: internal bibliography
%=====================================
%\begin{thebibliography}{999}
% Reference 1
%\bibitem[Author1(year)]{ref-journal}
%Author1, T. The title of the cited article. {\em Journal Abbreviation} {\bf 2008}, {\em 10}, 142--149.
% Reference 2
%\bibitem[Author2(year)]{ref-book}
%Author2, L. The title of the cited contribution. In {\em The Book Title}; Editor1, F., Editor2, A., Eds.; %Publishing House: City, Country, 2007; pp. 32--58.
%\end{thebibliography}

% The following MDPI journals use author-date citation: Arts, Econometrics, Economies, Genealogy, Humanities, IJFS, JRFM, Laws, Religions, Risks, Social Sciences. For those journals, please follow the formatting guidelines on http://www.mdpi.com/authors/references
% To cite two works by the same author: \citeauthor{ref-journal-1a} (\citeyear{ref-journal-1a}, \citeyear{ref-journal-1b}). This produces: Whittaker (1967, 1975)
% To cite two works by the same author with specific pages: \citeauthor{ref-journal-3a} (\citeyear{ref-journal-3a}, p. 328; \citeyear{ref-journal-3b}, p.475). This produces: Wong (1999, p. 328; 2000, p. 475)


%%%%%%%%%%%%%%%%%%%%%%%%%%%%%%%%%%%%%%%%%%
%% optional
%\sampleavailability{Samples of the compounds ...... are available from the authors.}

%% for journal Sci
%\reviewreports{\\
%Reviewer 1 comments and authors’ response\\
%Reviewer 2 comments and authors’ response\\
%Reviewer 3 comments and authors’ response
%}


